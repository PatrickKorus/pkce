\chapter{Reißverschlussverfahren}
\section{Reißverschlussverfahren - StVO}
\begin{center}
	\textit{"' Ist auf Straßen mit mehreren Fahrstreifen für eine Richtung das durchgehende Befahren eines Fahrstreifens nicht möglich oder endet ein Fahrstreifen, ist den am Weiterfahren gehinderten Fahrzeugen der Übergang auf den benachbarten Fahrstreifen in der Weise zu ermöglichen, dass sich diese Fahrzeuge \textbf{unmittelbar vor Beginn der Verengung jeweils im Wechsel} nach einem auf dem durchgehenden Fahrstreifen fahrenden Fahrzeug einordnen können (Reißverschlussverfahren)."'} - StVO \S 7 Absatz 4
\end{center}
\textbf{Vorteil:} - relativ einfach zu verstehendes Verfahren\\
- "faires Verfahren" die Spuren wechseln sich beim durchfahren der Engstelle ab, wodurch man auf beide Spuren ungefähr gleichschnell die Engstelle passiert\\
- die Fahrbahn wird bis zur Engstelle opimal genutzt\\
- das Verfahren hat sich in den letzen Jahren bewährt und jeder Autofahrer kennt die Theorie aus der Fahrschule\\
\textbf{Nachteil:} - viele Autofahrer sortieren sich schon weit vor Ende des Fahrstreifens ein\\
- bei großem Verkehrsaufkommen kann der Verkehr zum Stehen kommen\\
- Autofahrer müssen anhalten und andere Autofahrer reinzulassen\\

\section{Reißverschlussverfahren - Alternative}
\textbf{Idee:} Je höher die Geschwindigkeit desto mehr Autos können die Engstelle innerhalb eines Zeitraums passieren. Das alternative Reißverschluss verfolgt das Ziel, dass möglichst kein Auto stehenbeleiben muss, und somit kein Stau entsteht. Hierbei wird die Verengung bereits früher als normal angekündigt, damit das fahrbahnwechselne Auto genug Zeit hat sich eine Lücke zu suchen. Hierbei gilt die Regel, dass die erstbeste Lücke genutz wird. Hat das fahrbahnwechselne Auto eine geeignete Lücke gefunden sortiert es sich ein, indem es halb auf die andere Spur wechselt. Dadurch wird sichergestellt, dass Platz fürs Auto freigehalten wird und gleichzeitig verhindert das ein von hintenkommendes Auto noch überholt. \\
\textbf{Vorteil:} - die Autos sind bereits vor der Engstelle einsortiert, wodurch kein Auto zum stehen kommt\\
- der Verkehrsfluss wird nicht unterbrochen\\
\textbf{Nachteil:} - es werden nicht alle Lücken optimal genutzt\\
- bei hoher Verkehrsdichte kann es vorkommen das nicht genug Lücken vorhanden sind, bis es zur Engstelle kommt\\
- eine Änderung des Reißverschlussverfahren würden hohe Kosten verursachen, da man die neue Regel verbreiten muss\\
- nur bei Baustellen nutzbar, bei Hindernissen wie einen Unfall jedoch nutzlos keine frühzeitige Ankündigung möglich ist\\