\chapter{Programmierung}
\section{Attribute Fahrzeug}
\textbf{Schwierigkeit:} Welche Eigenschaften müssen einem Auto übergeben werden damit eine realistische Simulation möglich ist?
\begin{addmargin}[25pt]{0pt}
	\item \textbf{Lösungsansatz:} Beim Erstellen erhält das Auto eine Position und eine Geschwindigkeit. Die Position besteht aus einer Meteranzeige, wo auf der Strecke es sich aufhält und einer Abfrage auf welcher Spur es sich befindet. Zudem hat es die Möglichkeit zu beschleunigen bezeihungsweise zu bremsen und kann die zugelassene Höchstgeschwindigkeit abfragen. \\
\end{addmargin}

\section{Visualisierung}
\textbf{Schwierigkeit:} Wie lässt sich das Reißverschlussverfahren am besten simulieren?
\begin{addmargin}[25pt]{0pt}
	\item \textbf{Lösungsansatz:} Für die Simulation wurde Slick verwendet, da alle Gruppenmitlieder schon Erfahrungen mit Java gesammelt haben und Slick bereits eine graphische Oberfläche für Java besitzt, auf die zugegriffen werden konnte. Somit konnte gleich mit der Implementierung des Reißverschlussverfahren begonnen werden und es musste nicht erst eine graphische Umgebung erstellt werden. Um einen leichteren Einstieg in Slick zu erhalten,wurde ein bereits existierendes Spiel namens "'UfoInvasion"' benutzt, um zu überprüfen, ob Slick auf allen Computern läuft. Daraufhin wurde ein neues Programm geschrieben, in dem lediglich die Slick-Library übernommen wurde. \\
\end{addmargin}
\textbf{Schwierigkeit:} Wie sollte die Fahrbahn aufgebaut sein?
\begin{addmargin}[25pt]{0pt}
	\item \textbf{Lösungsansatz:} Wir betrachten eine Fahrbahnlänge von 1200\,m. Die Länge wurde gewählt, da sie lang genug ist, um die Auswirkungen des Reißverschlussverfahren zu beurteilen, zugleich aber nicht zu groß, damit man noch die einzelnen Autos erkennen kann. Sollte man einen kleineren Abschnitt betrachten wollen, gibt es auch die Möglichkeit heranzuzoomen.\\
\end{addmargin}
\textbf{Schwierigkeit:} Was für visuelle Eigenschaften muss das Auto besitzen?
\begin{addmargin}[25pt]{0pt}
	\item \textbf{Lösungsansatz:} Um die Handlung jedes einzelne Auto zu erkennen, haben wir den Autos Bremslichter und Blinker gegeben. Zudem zeigt die Fahrzeugfarbe den Fahrstyle an. Der aggressive Fahrer ist durch rosa, der durchschnittliche durch blau und der passive durch grün gekennzeichnet.\\
\end{addmargin}

\section{Spawner}
\textbf{Schwierigkeit:} Wie schafft man es einen realistischen Verkehrsfluss zu erzeugen, um den Eingangsverkehr in das Simulationsgebiet zu Simulieren.
%Wie schafft man es Fahrzeuge zu erzeugen, die mit einem realistischen Verkehrsverhalten an das Stauende fahren.
\begin{addmargin}[25pt]{0pt}
	\item \textbf{Lösungsansatz:} Hierzu muss eine Vielzahl von Aspekten beachtet werden. So zeichnet sich der Verkehrsfluss unter anderem dadurch aus, dass die einzelnen Autos genügend Abstand halten, sowie eine ähnliche (nicht exakt gleiche) Geschwindigkeiten halten. Ebenfalls wichtig ist es, die Verteilung auf die Fahrbahnen realitätsnah darzustellen und in diesem Zusammenhang die Verkehrsdichte variabel simulieren zu können.
	
	Um dies zu erreichen wurde als oberste Hürde die Verkehrsdichte(in \%) eingearbeitet. Hierzu wird eine normalverteilte Zufallszahl gewählt, wobei Erwartungswert und Standardabweichung umgekehrt proportional zur gewünschten Verkehrsdichte sind. Diese Zufallszahl gibt an, in welchem Intervall Autos erzeugt werden und wird nach jedem Auto neu berechnet. 
	
	ready for Spawn:
			-> safety_Dist \&\& Max_Spd
			-> lane free?
			-> spawn
	
	spawn: 
			-> 		
	%Bevor ein neues Fahrzeug gespawnt werden kann, müssen mehrere Bedingungen erfüllt sein. Dazu gehört zu aller erst der Sicherheitsabstand. Dieser wird aus der Geschwindigkeit des vorrausfahrenden Fahrzeuges und dem Tempolimit errechnet. Erst wenn gewährleistet ist, das genug Sicherheitsabstand existiert wird der Spawner freigegeben. Je nach gewünschter Verkehrsdichte spawnt dann nach einer gewissen Zeit ein neues Auto. Dieses Auto besitzt eine zufällige Geschwindigkeit. Diese Geschwindigkeit orientiert sich am vorrausfahrendem Fahrzeug. So liegt der Erwartungswert der Geschwindigkeit bei der Bahngeschwindigkeit des Vordermann plus ein Drittel der Differenz der Höchstgeschwindigkeit und der Bahngeschwindigkeit. Zudem ist geregelt, das Autos die auf der linke Spur erzeugt werden mindestens 80\% der aktuellen Bahngeschwindigkeit besitzen und auf der rechten mindestens 60\%. Dies verhindert das ein viel zu langsames Auto spawnt, was zum einen Realitätsfern wäre und zum Andern den Verkehrsfluss unnötigerweise verlangsamt.\\
\end{addmargin}

\section{Bedienung}
\textbf{Schwierigkeit:} Wie erhält man eine benutzerfreundliche Bedienungsoberfläche, bei der man möglichst viele Variablen verändern kann.
\begin{addmargin}[25pt]{0pt}
	\item \textbf{Lösungsansatz:} Die Steuerung über Eingabefelder hat sich als am Benutzerfreundlichsten herausgestellt. Hier hat der Benutzer die Möglichkeit auf Skalierung, Zeitraffer, Verkehrsdichte, sowie den Anteil der Fahrertypen, Einfluss zu nehmen. Um Fehler zu vermeiden, werden nur sinnvolle Eingaben berücksichtig.\\ 
	Mithilfe der Skalierung kann man an die Verengung herranzoomen. Bei der Standardeinstellung von 0.09 wird der komplette Simulationsbereich betrachten.\\ 
	Der Zeitraffer dient dazu Langzeitentlicklungen schneller darstellen zu können. Hierbei ist jedoch zu beachten, dass durch die Verwendung des Zeitraffers die Abtastrate sinkt. Dadurch ist die Genauigkeit nicht mehr gewährleistet. Bis zu einem Zeitraffer von 5 sind die Ergebnisse noch recht genau, bei einer schnelleren Betrachtung dient das Ergebnis lediglich als Annäherung.\\
	Durch die Verkehrsdichte kann man die Erzeugungswahrscheinlichkeit von Fahrzeugen verändern. Die Verkehrsdichte hat massgeblichen Einfluss auf die Effektivität des Reißverschlussverfahren. Diesen Einfluss kann der Benutzer anhand der ausgegebenen Durchschnittsgeschwindigkeit, der Eingangsverkehrsdichte, sowie der Ausgangsverkehrdichte ablesen.\\
	Zu guter letzt kann der Benutzer noch den Anteil der Fahrertypen variieren. Standardmäßig gibt es 33\% aggressive (rosa), 33\% passive (grün) und 33\% durchschnittliche Fahrer (blau). Die Anteile addieren sich immer zu 100\% auf, wobei die aggressiven und passiven Anteile durch Eingabefelder veränderbar sind und der durchschnittliche Fahrer die restlichen Anteile bekommt.\\
	Solle der Benutzer zusätzliche Informationen zu den jeweiligen Autos bekommen wollen, kann er durch Betätigung des "'D"' auf der Tastertur sich zu jedem Auto dessen ID, die derzeitige Geschwindigkeit, sowie die Beschleunigung anzeigen lassen. Diese Funktion wurde zum Debuggen erstellt, wodurch kein großer Wert auf die Lesbarkeit gelegt wurde und somit sich die Datenausgaben überlagern können.\\
\end{addmargin}

\section{Fahrer}
\textbf{Schwierigkeit:} Es gibt in der Realität viele unterschiedliche Fahrertype. Jeder Fahrer hat einen individuellen Fahrstyle. Diese lassen sich häufig in Kategorien wie zum Beispiel aggressive oder passiven Fahrer einsoritern. Zudem müsste man auch LKWs betrachtet werden. Ganz ohne die Unterscheidung von Fahrertypen wäre die Simulation nicht Realitätsgetreu.
\begin{addmargin}[25pt]{0pt}
	\item \textbf{Lösungsansatz:} Die Programmierung von sehr vielen Fahrertypen ist zeitaufwendig und alle psychologischen Aspekte sind nicht darstellbar. Wir haben uns für drei Fahrertypen entschieden: Den Aggressiven (rosa), den Durchschnittlichen (blau) und den Passiven(grün). Diese drei Fahrertypen unterscheiden sich darin, wie großen Sicherheitsabstand sie einhalten, mit welcher Geschwindigkeit sie erzeugt werden, wie stark der Fahrer abbremst sobald das vorrausfahrende Fahrzeug bremst und wie sie sich beim Einsortieren verhalten. So hat der aggressive Fahrer eine deutlich höhere Wahrscheinlichkeit, dass er mit einer erhöhten Geschwindigkeit erzeugt wird, als der passive oder der durchschnittliche Fahrer. Zudem nutzt der aggressive Fahrer jede noch so kleine Lücke die sich bei der Verengung ergibt. Der Passive hingegen lässt einen deutlich größenen Sicherheitsabstand zum vorrausfahrenden Fahrzeug und hat einen höheren "Panikfaktor", welcher bewirkt, dass er sobald vorne gebremst wird auch stark abbremst.\\
	Um die Simulation realitätsgetreuer zu machen besitzt jeder Fahrer über eine Reaktionszeit von 250/,ms. Dadurch bremsen nicht alle Fahrzeuge gleichzeitig sondern erst wenn der Vordermann auch bremst.\\
\end{addmargin}