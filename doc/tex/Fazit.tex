\chapter{Fazit}
Beide Verfahren haben ihre Vor- und Nachteile. Nach betrachten unterschiedlicher Verkehrsvorkommen sind wir zu dem Fazit gelangt, dass das Reißverschlussverfahren nach StVO, zurzeit das bessere Verfahren ist.
Das liegt daran, dass das alternativ getestete Reißverschlussverfahren bei hoher Verkehrsdichte seine Effektivität verliert. Da nicht jede Lücke optimal genutzt wird, kommt es zu einem gravierent längeren Rückstau, welcher gegebenenfalls zusätzlich Zufahrtsstraßen verstopft. Bei niedrigem Verkehrsvorkommen ist es jedoch, wie erwartet die bessere Methode. Da jedoch auf Autobahnen eher mit einer hohen Verkehrsdichte zu rechnen ist, haben diese Ergebnisse eine besonders hohe Gewichtung. Zudem ist das alternativ getestete Verfahren, wegen der Benötigten frühzeitigen Ankündigung, nur bei Baustellen nutzbar, während das jetztige Reißverschlussverfahren auch bei Hindernissen wie Unfällen gut einsetzbar ist.\\
Sollte es in ferner Zukunft zur Einführung von autonom fahrenden Fahrzeugen kommen, welche miteinander kommunizieren können, wäre das alternative Verfahren das Geeignetere. Dies liegt daran, dass der Beginn des Reißverschlussverfahren nicht durch Verkehrsschilder eingeleitet wird, sondern durch Kommunikation zwischen den Fahrzeugen, wodurch der Beginn des Einfädelungsprozess der Verkehrsdichte angepasst werden kann. Dann wäre jedoch die Frage ob sich das Auto selbst eine Lücke sucht oder ob es durch Kommunikation mit den anderern Fahrzeugen eine Lücke zugewiesen bekommt. 